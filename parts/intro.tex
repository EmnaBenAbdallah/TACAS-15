\section{Introduction} 

With both the spread of numerical tools in every part of daily life and the development of NGS methods (New Generation Sequencing methods), like DNA microarrays in biology, a large amount of time series data is now produced every day, every minute, every second \cite{marx2013biology}. This means that the number of experiments - and corresponding data - led on a biological system grows drastically. The newly produced data - as long as the associated noise does not raise an issue with regard to the precision and relevance of the corresponding information - can give us some new insights on the behavior of a system. This justifies the urge to design efficient revision methods that are able to update previous knowledge on a model with regard to additional information. In other words, there is a strong need for automatic methods that update a given model so that the dynamics of the model is consistent with given observations and a set of criteria (for example, minimize the number of modifications). 

Network completion has been the subject of numerous recent works. In \cite{akutsu2009completing}, the authors targeted the completion of stationary Boolean networks. This method has been further refined along the years. Latest works \cite{nakajima2013network} focus on completion in Time Varying Genetic Networks. These are networks whose topology does not change through time, but the nature of the interactions (activation, inhibition, or no interaction) between components may change at some (finite number of) time points. The completion approach (which, in these authors' papers, refers to both addition and deletion of interactions, making it a synonym of revision) has been successfully applied to biological case-studies, for example the DREAM4 Challenge \cite{nakajima2014network}, and the implementation has been improved through heuristics \cite{nakajima2014exact}. The method however is limited to acyclic networks. 

Logical-based approaches may be also fruitful to network revision. It has been successfully applied to causal networks \cite{inoue2013completing} and molecular networks represented with the SBGN-AF language \cite{yamamoto2014completing}. 

Despite being a proper research area, revision is strongly connected to model inference. Starting from an empty model, revision may indeed be used to model from gene expression data. On the reverse, existing inference algorithms may be an inspiration for new revision methodologies. In particular, Answer Set Programming (ASP), a form of declarative programming that has been successively used in many knowledge representation and reasoning tasks \cite{DBLP:journals/amai/Niemela99,Baral03,DBLP:conf/iclp/Baral08} has been proven useful for network reconstruction \cite{durzinsky2011automatic} and inference of metabolic networks \cite{videla2014learning}. 
%	\emph{Answer set programming} (ASP) is a form of declarative programming that has been successively used in many knowledge representation and
%	reasoning tasks \cite{DBLP:journals/amai/Niemela99,Baral03,DBLP:conf/iclp/Baral08}.
%	In ASP, a problem is represented by a logic program where the answer sets correspond to the solutions of the problem.
%	Solving the problem is then reduced to computing stable models using answer set solvers like \emph{clasp} \cite{DBLP:conf/lpnmr/GebserKNS07a,gebser2008user}.

To our knowledge, no works have been led so far in the field of revision of timed models, without any restriction on the structure of the network. 

In this paper, we aim to provide a logical approach to tackle the revision of qualitative models of biological dynamic systems, like gene regulatory networks. In our context, we assume the set of interacting components as fixed and we consider potential additions/deletions of interactions between components. The main originality of our work is that we address this problem in a timed setting, with quantitative delays potentially occurring between the moment an interaction activated and the moment its effect is visible. It allows for example to catch delays between the activation of a gene, and the moment the concentration of a gene reaches a qualitative threshold. 

During the past decade, there has been a growing interest for the hybrid modeling of gene regulatory networks with delays. These hybrid approaches consider various modeling framerworks. In \cite{matsuno2000hybrid}, the authors hybrid Petri nets: the advantage of hybrid with regard to discrete modeling lies in the possibility of capturing biological factors, e.g., the delay for the transcription of RNA polymerase. The merits of other hybrid formalisms in biology have been studied, for instance timed automata \cite{siebert2008temporal} and hybrid automata \cite{ahmad2006hybrid}. 
Finally, in \cite{comet2010formal}, the authors investigate a direct extension of the discrete Ren\'e Thomas' modeling approach by introducing quantitative delays. These delays represent the compulsory time for a gene to turn from a discrete qualitative level to the next (or previous) one. They exhibit the advantage of such a framework for the analysis of mucus production in the bacterium Pseudomonas aeruginosa. The approach we propose in this paper inherits from this idea that some models need to capture these timing features. 

In order to address the formal checking of dynamical properties within very large BRNs, we previously introduced in \cite{PMR10-TCSB} a new formalism, named the \emph{``Process Hitting''} (PH), to model concurrent systems having components with a few qualitative levels. Being a particular restriction of asynchronous automata networks or safe Petri nets, Process Hitting can be applied to complex dynamical systems with a very large number of interacting components, where each of these components can be described with a few internal states. In this paper, following recent works enriching (by adding priorities) the expressivity of PH while preserving its efficiency \cite{folschette2013under}, we extend PH with quantitative timing features and exhibit efficient ASP-based approaches to perform network revision. 

As readers may not be familiar with PH, we briefly introduce it in \pref{sec:ph} and all theoretical and practical notions are then settled to introduce our timed extension of PH. Then in  \pref{sec:ph-generation} we present the related generation algorithm and we demonstrate the performance of our algorithme by a case study runnig example. We propose in \ref{sec:refinement-filters} as well new refinement methods applied to the generated model. Then we illustrate the merits of our approach in section \ref{sec:evaluation} by first applying it on dynamical biological models, then discussing the practical results on a range of benchmarks from the DREAM4 datasets, a popular reverse-engineering challenge, and discuss the computational performances of our algorithm. Finally, in section \ref{sec:conclusion}, we summarize our contribution and give some perspectives for future works. 


