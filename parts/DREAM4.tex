\section{Evaluation}
	\label{sec:evaluation}
	
	In this section, we assess the efficiency of our algorithm through case studies coming from the DREAM4 challenge \cite{prill2011crowdsourcing}.

	DREAM challenges are annual reverse engineering challenges that provide biological case studies.
	In this paper, we focus on the datasets coming from DREAM4.
	The input data that we tackle here consists of the following:
	5 different systems each composed of 10 genes, all coming from E. coli and yeast networks. For every such system,
	the available data are the following: (i) 5 time series data with 21 time points; (ii) steady state at wild type;
	(iii) steady states after knocking out each gene;
	(iv) steady states after knocking down each gene (i.e. forcing its transcription rate at 50\%);
	(v) steady states after some random multifactorial perturbations. We processed all the data.
	Here, we focus on the management of time series data.

\subsection{Settings}

	Time series data provide us 20 transitions.
	Each of them include different perturbations that are maintained all time along during the first 10 transitions and applied to at most 3 genes.
	In this setting, a perturbation means a significant increase or decrease of the gene expression.
	%
	In the raw data of the time series, gene expression values are given as real number between 0 and 1.
	To apply our approach, we chose to discretize those data into 4 qualitative values.
	Each gene is discretized in an independent manner, with respect to the following procedure:
	we compute the average value of the gene expression among all data of a time series,
	then the values between the average and the maximal/minimal value are divided into as many levels.
	Discretizing the data according to the average value of expression is expected to reduce the impact of perturbation on the discretization and thus on the model learned.

\subsection{Results}
