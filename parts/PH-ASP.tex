% PH through ASP

\section{PH through ASP}
\label{sec:ph-asp}
\subsection{Answer Set Programming}

	In this section, we recapitulate the basic elements of ASP.
	An answer set program is a finite set of rules of the form
	\begin{equation}
		\label{rule}
		a_{0}\ \emph{:-}\ a_{1},\ \ldots,\ a_{m},\ not\ a_{m+1},\ \ldots,\ not\ a_{n}\
	\end{equation}
	where $n \ge m \ge 0$, $a_{0}$ is a propositional atom or $\bot$, all
	$a_{1}, \ldots ,a_{n}$ are propositional atoms and the symbol "$not$" denotes default negation.
	If $a_{0} = \bot$, then Rule (\ref{rule}) is a constraint (in which case $a_{0}$ is usually omitted).
	The intuitive reading of a rule of form (\ref{rule}) is that whenever $a_{1}, \ldots, a_{m}$
	are known to be true and there is no evidence for any of the default negated atoms $a_{m+1}, \ldots, a_{n}$ to be true, then $a_{0}$ has to be true as well.
	Note that $\bot$ can never become true.
	
	In the ASP paradigm, the search of solutions consists in computing answer sets of answer set program.
	An answer set for a program is defined by Gelfond and Lifschitz \cite{DBLP:conf/iclp/GelfondL88} as follows.
	An interpretation $I$ is a finite set of propositional atoms.
	An atom $a$ is true under $I$ if $a \in I$, and false otherwise.
	A rule $r$ of form \ref{rule} is true under $I$ if $\{a1,\ \dots,\ a_{m}\} \subseteq I$ and $\{a_{m+1},\ \ldots,\ a_{n}\} \cap I = \emptyset$ implies $a_{0} \in\ I$.
	An Interpretation $I$ is a model of a program $P$ if each rule $r \in P$ is true under $I$.
	Finally, $I$ is an answer set of $P$ if $I$ is a subset-minimal model of $P^{I}$,
	where $P^{I}$ is defined as the program that results from $P$ by deleting all rules that contain a default negated atom from $I$, 
	and deleting all default negated atoms from the remaining rules.
	Programs can yield no answer set, one answer set, or many answer sets.
	To compute answer sets of an answer set program, we run an ASP solver.

\subsection{Translation of PH networks to ASP}
PH network is easy to be formulated in ASP. Indeed we need only three predicates to define the whole network:
\texttt{"sort"} to define sorts, \texttt{"process"} for the processes and \texttt{"action"} for the network actions.
In \cite{MOVEP14}, we already showed the interest of ASP regarding Process Hitting (without plural actions, nor delays). 
We will see in example \ref{ex1:asp-ph} how a PH network is defined with these predicates.


\begin{example}[Representation of a PH network in ASP]
\label{ex1:asp-ph}
The representation of the PH given in figure \ref{fig:ph} into ASP is the following:
\begin{lstlisting}
process("a", 0..1). process("b", 0..1). process("z", 0..2). %\label{ASPprocess}
sort(X) :- process(X,I) %\label{ASPsort}
action("a",0,"b",1,0). action("a",1,"a",1,0). action("b",1,"z",0,2). %\label{actions1}
action("b",0,"z",1,2). action("z",0,"a",0,1). %\label{actions2}
\end{lstlisting}
In line \ref{ASPprocess}, we create the list of processes corresponding to each sort,
for example the sort \texttt{"z"} has 3 processes numbered from \texttt{0} to \texttt{2};
this specific predicate will in fact expand into the three following predicates:
\texttt{process("z", 0)}, \texttt{process("z", 1)}, \texttt{process("z", 2)}.
Line \ref{ASPsort} enumerates every sort of the network from the previous information.
Finally, all actions of the model are defined in lines \ref{actions1} and \ref{actions2};
for example, the first predicate \texttt{action("a",0,"b",1,0)} represents the action
$\PHfrappe{a_0}{b_1}{b_0}$.
\end{example}

\begin{example}[Representation of PH with plural-timed actions in ASP]
\label{ex2:ph-asp}
We can take the example given in figure \ref{fig:ph-plurial} and consider that the action  $\{x_1, y_1, z_0 \} \xrightarrow{D} \{x_1, y_1, z_1 \} $  has a delay "$D$" and in PH the action becomes: $x_1 \wedge \PHfrappe{y_1}{z_0}{z_1}$. As a consequence, the PH translated in ASP is the following:
\begin{lstlisting}
process("x", 0..1). process("y", 0..1). process("z", 0..1). %\label{ASPprocess2}
action("x",1,"y",1,"z",0,1,D). %\label{pluralaction}
\end{lstlisting}
The number of indegree in this action $i=2$, there is only 2 hitters. It is possible to have an indegree greater than $2$. We will show later that the number of the maximum indegree should be an input for our algorithm.
\end{example}

The predicate \texttt{action} represents the ordinary actions as well as the plural actions. Indeed an ordinary action is a plural action with an indegree $1$. Moreover all the ordinary actions are timed actions with a delay equal to $1$ unit of time. Every action needs at least one step to be played. For example, the action \texttt{action("a",0,"b",1,0)} is equivalent to \texttt{action("a",0,"b",1,0,1)}
As a result, the principle of having plural-timed actions in the modeling sounds like a generalized way to represent the actions in a Process Hitting network. That is why, in the following, we will consider only plural-timed actions. 
	
