\section{Answer Set Programming}
\label{sec:asp}
	In this section, we recapitulate the basic elements of ASP.
	An answer set program is a finite set of rules of the form
	\begin{equation}
		\label{rule}
		a_{0}\ \emph{:-}\ a_{1},\ \ldots,\ a_{m},\ not\ a_{m+1},\ \ldots,\ not\ a_{n}\
	\end{equation}
	where $n \ge m \ge 0$, $a_{0}$ is a propositional atom or $\bot$, all
	$a_{1}, \ldots ,a_{n}$ are propositional atoms and the symbol "$not$" denotes default negation.
	If $a_{0} = \bot$, then Rule (\ref{rule}) is a constraint (in which case $a_{0}$ is usually omitted).
	The intuitive reading of a rule of form (\ref{rule}) is that whenever $a_{1}, \ldots, a_{m}$
	are known to be true and there is no evidence for any of the default negated atoms $a_{m+1}, \ldots, a_{n}$ to be true, then $a_{0}$ has to be true as well.
	Note that $\bot$ can never become true.
	
	In the ASP paradigm, the search of solutions consists in computing answer sets of answer set program.
	An answer set for a program is defined by Gelfond and Lifschitz \cite{DBLP:conf/iclp/GelfondL88} as follow.
	An interpretation $I$ is a finite set of propositional atoms.
	An atom $a$ is true under $I$ if $a \in I$, and false otherwise.
	A rule $r$ of form \ref{rule} is true under $I$ if $\{a1,\ \dots,\ a_{m}\} \subseteq I$ and $\{a_{m+1},\ \ldots,\ a_{n}\} \cap I = \emptyset$ implies $a_{0} \in\ I$.
	An Interpretation $I$ is a model of a program $P$ if each rule $r \in P$ is true under $I$.
	Finally, $I$ is an answer set of $P$ if $I$ is a subset-minimal model of $P^{I}$,
	where $P^{I}$ is defined as the program that results from $P$ by deleting all rules that contain a default negated atom from $I$, 
	and deleting all default negated atoms from the remaining rules.
	Programs can yield no answer set, one answer set, or many answer sets.
	To compute answer sets of an answer set program, we run an ASP solver.
