\section{Refinement of genrated PH model }
The Algorithm \ref{alg:PHG_ap} returns a PH model with a set of actions compeletely coherent with the observations as well as the regulatory genes influences.
But in practice this set of actions can be further refined using a background knowledge.
The following filtering operation can help to reduce the complexity of the model learned/revised.

\subsection{$1^{st}$ filter: priority on given model}

If we want to minimize the number of actions added to the input PH  model ($PH_{ini}$)we can consider that the actions of this PH are more trustable than the generated one. Thus generated action(s) explaining a change can already be explained by a given action in $PH_{ini}$. So that the generated action(s) can be discarded:\\
\begin{definition}[Filter: Priority on given model]
Let $T$ be the set of time steps of the given time series data, we have $\forall t \in T$ such that $\exists$  \texttt{change(t)}, we have:
\begin{itemize}
\item[-] \texttt{$H_c$}: the set of candidate actions generated to explain the \texttt{change(t)}
\item[-] \texttt{$H_{ini}$:} the set of actions of $PH_{ini}$ that can realize the \texttt{change(t)}
\item[-] \texttt{$H_{final}$:} the set of actions to keep to model the \texttt{change(t)}
\end{itemize}
We have either:
\begin{itemize}
\item[--] If $H_c \cap H_{ini} \neq \emptyset $ then $H_{final}= H_c \cap H_{ini}$ 
\item[--] If $H_c \cap H_{ini} = \emptyset $ then $H_{final}=H_c$
\end{itemize}
\end{definition}
In practice, the change that already can be explained by the input PH can be detected before computing the candidates actions, allowing us to discard them without generating them.

If in our running example (\pref{sec:case_study}),
we start with $H_{ini}=\{h_{ini}=\PHfrappedelay{a_0}{1}{z_1}{z_0}\}$,
since the action $h_{ini}$ can realize $change(6)$ there is no need to generate new ones.
Here, $H_{change(6)}$ will only consist of $h_{ini} = h_6$. So $h_7=\PHfrappedelay{b_0}{1}{z_1}{z_0}$ and $h_8=\PHfrappedelay{a_0 \wedge b_0 }{1}{z_1}{z_0} $ will be discarded. 

\subsection{$2^{nd}$ filter: strict influences (activator or inhibitor)}

If knowledge about strict influences is given it can be used to discard actions that are conflicting with those influences.
For example, if we know that a gene $"a"$ influences a gene $"b"$ and that $"a"$ can only inhibit $"b"$, then the actions using $"a"$ as an active hitter to increase the value of $"b"$ are inconsistent and can be discarded.

\begin{definition}[Filter: Strict influences]
Let $PH=(\PHs,\PHl,\PHa_p)$ be the input model, $\forall h \in \PHa_p$ such $h=\PHfrappedelay{A}{D}{b_n}{b_m}$, with $ A \in \PHl^{\diamond}, | A| \leq i$ ($i$: indegree of the model) we have :\\
If $n < m$ (resp. $n > m$ ): if $\exists G_k \in A $ with $G_k \in \PHl_G $ such that $G \xrightarrow{(-)} b$ and $\nexists G \xrightarrow{(+)} b$ (resp. $G \xrightarrow{(+)} b$ and $\nexists G \xrightarrow{(-)} b$) and $k \neq 0$ then $h$ can be discarded.
\end{definition}

In our running example (\pref{sec:case_study}), if we know that $"b"$ is an inhibitor of $"z"$, all actions where $"b"$ is used to activate $"z"$ can be discarded.
Here, we have an observation where $"z"$ is activated in $change(2)$ where its value switches from $z_0$ to $z_1$.
We know that $"a"$ has an influence on $"z"$ but as an inhibitor not as an activator thus the actions $h_2=\PHfrappedelay{b_1}{2}{z_0}{z_1}$ and $h_3=\PHfrappedelay{a_0 \wedge b_1 }{2}{z_0}{z_1}$ will be discarded.

\subsection{$3^rd$ filter: delay merging}

When the time information of observation is not perfect, the same regulation interaction may append with different delay.
One simple solution to deal with such input can be to simply agregate action that differ only by their delays.
We can merge each action with the same hitters, $S_1,P_1,\ldots, S_n,P_n$ the same target, $G, P$ and the same bound $G, P'$ into one action where the delay is the average.

\begin{definition}[Filter: Delay merging]
$\forall h_1, h_2,..., h_k \in H$ such that $h_1=\PHfrappedelay{A}{D_1}{a_n}{a_m}$, $h_2=\PHfrappedelay{A}{D_2}{a_n}{a_m}$, ..., $h_k=\PHfrappedelay{A}{D_k}{a_n}{a_m}$ with $ A \in \PHl^{\diamond}$, $a_n, a_m \in \PHl_a$ and $D_1 \neq D_2 \neq ... \neq D_k$ then : \\
\texttt{fusion} all actions $h_1, h_2,..., h_k$ into one action $h$: $$h=\PHfrappedelay{A}{D_{average}}{a_n}{a_m}, \hspace{0.7cm} \text{such that,} \hspace{0.7cm} D_{average} = \frac{\sum_{i=1}^k D_{i}}{k} $$
\end{definition}

For example, lets suppose that we came out with two actions $h_1=\PHfrappedelay{a_0}{2}{z_0}{z_1}$ and $h_2=\PHfrappedelay{a_0 \wedge b_1 }{4}{z_0}{z_1}$, they will be merged into $h=\PHfrappedelay{b_1}{3}{z_0}{z_1}$.
If input data are perfect, usage of this merging is totally unsafe and can lead to a set of actions that cannot produce any of the observed changes in the worst case.
The intuition behind this method is to give a first idea of how to cope with big amount of real data which are not perfect.
The idea is that if enough observations are provided, the delay of the action will be more precise.
