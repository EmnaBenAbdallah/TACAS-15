
%%%%%%%%%%%%%%%%%%%%%%% file main.tex %%%%%%%%%%%%%%%%%%%%%%%%
%
% Article à déposer pour TACAS 2015
%
%%%%%%%%%%%%%%%%%%%%%%%%%%%%%%%%%%%%%%%%%%%%%%%%%


\documentclass[runningheads,a4paper]{llncs}

\usepackage{amssymb}
\setcounter{tocdepth}{3}
\usepackage{graphicx}

\usepackage{url}

%%%%%%%%%%%%%%%%%% Packages %%%%%%%%%%%%%%%%%%%%%%%%%
\usepackage{hyperref}


\usepackage{amsmath}  % Maths
\usepackage{amsfonts} % Maths
\usepackage{amssymb}  % Maths
\usepackage{stmaryrd} % Maths (crochets doubles)
\usepackage{ dsfont }

\usepackage{algorithm}
\usepackage{algpseudocode}


%%%%%%%%%%%%%%%%%%%%
\usepackage{listings}
% Définition du langage ASP
\lstdefinelanguage{ASP}{\^^M}
}

% Définition des styles de tous les listings du document
\lstset{language=ASP,
basicstyle=\small,
columns=flexible,
keywordstyle=\bfseries,
firstnumber=last
}
\renewcommand{\thelstnumber}{\the\value{lstnumber}}
%% fin définition

%%%%%%%%%%%%%%%%%%%%%%%%%%%%%%%%%%
\usepackage{enumerate} % Personnalisation de la numérotation des listes
\usepackage{url}     % Mise en forme + liens pour URLs
\usepackage{array}   % Tableaux évolués
\usepackage{comment}
\usepackage{moresize}
\usepackage{setspace}

\usepackage{prettyref}
\newrefformat{def}{Definition~\ref{#1}}
\newrefformat{fig}{Figure~\ref{#1}}
\newrefformat{pro}{Property~\ref{#1}}
\newrefformat{pps}{Proposition~\ref{#1}}
\newrefformat{lem}{Lemma~\ref{#1}}
\newrefformat{th}{Theorem~\ref{#1}}
\newrefformat{sec}{Section~\ref{#1}}
%\newrefformat{subsec}{Subsect.~\ref{#1}}
\newrefformat{suppl}{Appendix~\ref{#1}}
\newrefformat{eq}{Eq.~\eqref{#1}}
\def\pref{\prettyref}


\usepackage{tikz}
\newdimen\pgfex
\newdimen\pgfem
\usetikzlibrary{arrows,shapes,shadows,scopes}
\usetikzlibrary{positioning}
\usetikzlibrary{matrix}
\usetikzlibrary{decorations.text}
\usetikzlibrary{decorations.pathmorphing}

%\input{macros/macros}

%%%%%
% Macros générales
\def\Pint{\textsc{PINT}}


% Notations spécifiques à ce papier
\newcommand{\PHdirectpredec}[1]{\PHs^{-1}(#1)}
\newcommand{\PHpredec}[1]{\f{pred}(#1)}
\newcommand{\PHpredecgene}[1]{\f{reg}({#1})}
\newcommand{\PHpredeccs}[1]{\PHpredec{#1} \setminus \Gamma}

\tikzstyle{boxed ph}=[]
\tikzstyle{sort}=[fill=lightgray,rounded corners]
\tikzstyle{process}=[circle,draw,minimum size=15pt,fill=white,
font=\footnotesize,inner sep=1pt]
\tikzstyle{black process}=[process, fill=black,text=white, font=\bfseries]
\tikzstyle{gray process}=[process, draw=black, fill=lightgray]
\tikzstyle{current process}=[process, draw=black, fill=lightgray]
\tikzstyle{process box}=[white,draw=black,rounded corners]
\tikzstyle{tick label}=[font=\footnotesize]
\tikzstyle{tick}=[black,-]%,densely dotted]
\tikzstyle{hit}=[->,>=angle 45]
\tikzstyle{selfhit}=[min distance=30pt,curve to]
\tikzstyle{bounce}=[densely dotted,->,>=latex]
\tikzstyle{hl}=[font=\bfseries,very thick]
\tikzstyle{hl2}=[hl]
\tikzstyle{nohl}=[font=\normalfont,thin]


\tikzstyle{aS}=[every edge/.style={draw,->,>=stealth}]
\tikzstyle{Asol}=[draw,circle,minimum size=5pt,inner sep=0,node distance=1.5cm]
\tikzstyle{Aproc}=[draw,node distance=1.2cm]
\tikzstyle{Aobj}=[node distance=1.5cm]
\tikzstyle{Anos}=[font=\Large]

%\tikzstyle{AprocPrio}=[Aproc,double]
\tikzstyle{AsolPrio}=[Asol,double]
\tikzstyle{AprocPrio}=[Aproc,double]
\tikzstyle{aSPrio}=[aS,double]

% Commandes À FAIRE
%\usepackage{color} % Couleurs du texte
%\newcommand{\todo}[1]{\textcolor{red}{\textbf{[[#1]]}}}
%\newcommand{\TODO}{\todo{TODO}}

%%%%%
% Id est
%\newcommand{\ie}{\textit{i.e.} }
\newcommand{\ie}{i.e.\ }
\newcommand{\resp}{resp.\ }

% Césures
\hyphenation{pa-ra-me-tri-za-tion}
\hyphenation{pa-ra-me-tri-za-tions}

\input{macros/macros}
\input{macros/macros-ph}
\input{macros/tikzstyles2.tex}
\input{macros/macros-abstr}

%%%%%%%%%%%%%%%%%%%%%%%%%%%%%%%%%%


\urldef{\mailsa}\path|{emna.ben-abdallah, morgan.magnin, olivier.roux}@irccyn.ec-nantes.fr|    
\urldef{\mailsb}\path|{tony_ribeiro, inoue}@nii.ac.jp|    
\newcommand{\keywords}[1]{\par\addvspace\baselineskip
\noindent\keywordname\enspace\ignorespaces#1}

\begin{document}

\mainmatter  % start of an individual contribution

\title{Generation and revision of Delayed Biological Regulatory Systems}


% a short form should be given in case it is too long for the running head
\titlerunning{Generation and revision of Delayed Biological Regulatory Systems}


\author{Emna Ben Abdallah\inst{1} \and  Tony Ribeiro\inst{1}  \and Morgan Magnin\inst{1,2} \and Olivier Roux\inst{1} \and Katsumi Inoue \inst{2}}
%
\authorrunning{E. Ben Abdallah, T. Ribeiro, M. Magnin, O.Roux, K. Inoue}
% (feature abused for this document to repeat the title also on left hand pages)

% the affiliations are given next; don't give your e-mail address
% unless you accept that it will be published
\institute{LUNAM Universit\'e, \'Ecole Centrale de Nantes,
 IRCCyN UMR CNRS 6597\\ (Institut de Recherche en Communications et Cybern\'etique de Nantes), \\
  1 rue de la No\"{e}, 44321 Nantes, France. \\
\and
National Institute of Informatics, \\
2-1-2, Hitotsubashi, Chiyoda-ku, Tokyo 101-8430, Japan.
}

%\mailsa\\
%\mailsb

%
% NB: a more complex sample for affiliations and the mapping to the
% corresponding authors can be found in the file "llncs.dem"
% (search for the string "\mainmatter" where a contribution starts).
% "llncs.dem" accompanies the document class "llncs.cls".
%

\toctitle{Revision of delayed biological regulatory systems}
\tocauthor{Authors' Instructions}
\maketitle


\begin{abstract}
The modeling of biological systems relies on background knowledge, deriving either from literature and/or the analysis of biological observations. But with the development of high-throughput data, there is a growing need for methods that automatically generate (resp. revise) admissible models with regard to initial (resp. additional) observations provided by biologists. 
Our research aims at providing a logical approach to generate (resp. to revise) biological regulatory networks thanks to time series data, i.e., gene expression data depending on time. In this paper, we propose a new methodology for models expressed through a timed extension of the Process Hitting framework (which is a restriction, well suited for biological systems, of networks of synchronized timed automata). The revision methods we introduce aim to make the minimum amount of modifications (addition/deletion of actions between biological components) to a given input model so that the resulting model is the most consistent as possible with the observed data. The originality of our work relies on the integration of quantitative time delays directly in our learning approach. 
Finally, we show the benefits of such automatic approach on dynamical biological models. In order to exhibit the scalability of our contribution, we conduct benchmarks on the DREAM4 datasets, a popular reverse-engineering challenge, and discuss the computational performances of our algorithm. 

\keywords{network construction/revision, dynamic modeling, delayed biological regulatory networks, time-varying genetic networks, Process Hitting}
\end{abstract}

  \section{Introduction} 

With both the spread of numerical tools in every part of daily life and the development of new measurement technologies (like DNA microarrays in biology), a large amount of time series data is now produced every day, every minute, every second \cite{marx2013biology}. This means that the number of experiments - and corresponding data - led on a biological system grows drastically. The newly produced data - as long as the associated noise does not raise an issue with regard to the precision and relevance of the corresponding information - can give us some new insights on the behavior of a system. This justifies the urge to design efficient revision methods that are able to update previous knowledge on a model with regard to additional information. In other words, there is a strong need for automatic methods that update a given model so that the dynamics of the model is consistent with given observations and a set of criteria (for example, minimize the number of modifications). 

Network completion has been the subject of numerous recent works. In \cite{akutsu2009completing}, the authors targeted the completion of stationary Boolean networks. This method has been further refined along the years. Latest works \cite{nakajima2013network} focus on completion in Time Varying Genetic Networks. These are networks whose topology does not change through time, but the nature of the interactions (activation, inhibition, or no interaction) between components may change at some (finite number of) time points. The completion approach (which, in these authors' papers, refers to both addition and deletion of interactions, making it a synonym of revision) has been successfully applied to biological case-studies, for example the DREAM4 Challenge \cite{nakajima2014network}, and the implementation has been improved through heuristics \cite{nakajima2014exact}. The method however is limited to acyclic networks. 

Logical-based approaches may be also fruitful to network revision. It has been successfully applied to causal networks \cite{inoue2013completing} and molecular networks represented with the SBGN-AF language \cite{yamamoto2014completing}. 

Despite being a proper research area, revision is strongly connected to model inference. Starting from an empty model, revision may indeed be used to model from gene expression data. On the reverse, existing inference algorithms may be an inspiration for new revision methodologies. In particular, Answer Set Programming (ASP), a form of declarative programming that has been successively used in many knowledge representation and reasoning tasks \cite{DBLP:journals/amai/Niemela99,Baral03,DBLP:conf/iclp/Baral08} has been proven useful for network reconstruction \cite{durzinsky2011automatic} and inference of metabolic networks \cite{videla2014learning}. 
%	\emph{Answer set programming} (ASP) is a form of declarative programming that has been successively used in many knowledge representation and
%	reasoning tasks \cite{DBLP:journals/amai/Niemela99,Baral03,DBLP:conf/iclp/Baral08}.
%	In ASP, a problem is represented by a logic program where the answer sets correspond to the solutions of the problem.
%	Solving the problem is then reduced to computing stable models using answer set solvers like \emph{clasp} \cite{DBLP:conf/lpnmr/GebserKNS07a,gebser2008user}.

To our knowledge, no works have been led so far in the field of revision of timed models, without any restriction on the structure of the network. 

In this paper, we aim to provide a logical approach to tackle the revision of qualitative models of biological dynamic systems, like gene regulatory networks. In our context, we assume the set of interacting components as fixed and we consider potential additions/deletions of interactions between components. The main originality of our work is that we address this problem in a timed setting, with quantitative delays potentially occurring between the moment an interaction activated and the moment its effect is visible. It allows for example to catch delays between the activation of a gene, and the moment the concentration of a gene reaches a qualitative threshold. 

During the past decade, there has been a growing interest for the hybrid modeling of gene regulatory networks with delays. These hybrid approaches consider various modeling framerworks. In \cite{matsuno2000hybrid}, the authors hybrid Petri nets: the advantage of hybrid with regard to discrete modeling lies in the possibility of capturing biological factors, e.g., the delay for the transcription of RNA polymerase. The merits of other hybrid formalisms in biology have been studied, for instance timed automata \cite{siebert2008temporal} and hybrid automata \cite{ahmad2006hybrid}. 
Finally, in \cite{comet2010formal}, the authors investigate a direct extension of the discrete Ren\'e Thomas' modeling approach by introducing quantitative delays. These delays represent the compulsory time for a gene to turn from a discrete qualitative level to the next (or previous) one. They exhibit the advantage of such a framework for the analysis of mucus production in the bacterium Pseudomonas aeruginosa. The approach we propose in this paper inherits from this idea that some models need to capture these timing features. 

In order to address the formal checking of dynamical properties within very large BRNs, we previously introduced in \cite{PMR10-TCSB} a new formalism, named the \emph{``Process Hitting''} (PH), to model concurrent systems having components with a few qualitative levels. Being a particular restriction of asynchronous automata networks or safe Petri nets, Process Hitting can be applied to complex dynamical systems with a very large number of interacting components, where each of these components can be described with a few internal states. In this paper, following recent works enriching (by adding priorities) the expressivity of PH while preserving its efficiency \cite{folschette2013under}, we extend PH with quantitative timing features and exhibit efficient ASP-based approaches to perform network revision. 

As readers may not be familiar with PH, we briefly introduce it in section \ref{sec:ph}, then give in section \ref{sec:ph-asp} some preliminary insights about recent translation of PH into ASP presented in \cite{benabdallah2015}. All theoretical and practical notions are then settled to introduce our timed extension of PH, and related completion algorithm in section \ref{sec:ph-completion}. Then we illustrate the merits of our approach in section \ref{sec:evaluation} by first applying it on a simplified model of mammalian circadian clock \cite{comet2012simplified}, then discussing the practical results on a range of benchmarks from bioinformatics literature. Finally, in section \ref{sec:conclusion}, we summarize our contribution and give some perspectives for future works. 



  % Process Hitting

\section{Process Hitting}
\label{sec:ph}

Definition \ref{def:PH} introduces the Process Hitting (PH) framework \cite{PMR10-TCSB}
which allows to model a finite number of local levels,
called \emph{processes},
grouped into a finite set of components, called \emph{sorts}.
A process is noted $a_i$, where $a$ is the sort's name,
and $i$ is the process identifier within sort $a$.
At any time, exactly one process of each sort is \emph{active},
and the set of active processes is called a \emph{state}.

The concurrent interactions between processes are defined by a set of \emph{actions}.
Each action is responsible for the replacement of one process by another of the same sort
conditioned by the presence of at least one other process in the current state.
A normal action is denoted by $\PHfrappe{a_i}{b_j}{b_k}$, which is read as
“$a_i$ \emph{hits} $b_j$ to make it \emph{bounce} to $b_k$”,
where $a_i$, $b_j$, $b_k$ are processes of sorts $a$ and $b$,
called respectively \emph{hitter}, \emph{target} and
\emph{bounce} of the action. 
We also call a \emph{self-hit} any action whose hitter and target sorts are the same,
that is, of the form: $\PHfrappe{a_i}{a_i}{a_k}$. The original Process Hitting framework contains only actions with one hitter but it should be noted that during these last years it was gradually enriched with new type of sorts like cooperative sorts and new actions like plural actions \cite{folschette-phd-14} (at least 2 hitters), actions with priority \cite{FPMR13-CS2Bio} and actions with delay.

The PH is therefore a restriction of asynchronous automata, where each transition
changes the local state of exactly one automaton,
and is triggered by the local states of at most two distinct automata.
This restriction in the form of the actions was chosen to permit
the development of efficient static analysis methods
based on abstract interpretation \cite{PMR12-MSCS}.

\begin{definition}[Process Hitting]\label{def:PH}
  A \emph{Process Hitting} is a triple $(\PHs,\PHl,\PHa_p)$ where:
  \begin{itemize}
    \item  $\PHs = \{a,b,\dots\}$ is the finite set of \emph{sorts};
    \item  $\PHl = \prod_{a\in\PHs} \PHl_a$ is the set of \emph{states} where
      $\PHl_a = \{a_0,\dots,a_{l_a}\}$
      is the finite set of \emph{processes} of sort $a\in\Sigma$
      and $l_a$ is a positive integer, with $a\neq b\Rightarrow \PHl_a \cap \PHl_b = \emptyset$;
    \item $\PHa_p$ = \{ $\PHfrappe{A}{b_j}{b_k}$ with $A \in \PHl^{\diamond} \wedge b \in \PHs \wedge b_j \neq b_k \wedge$ if $b_j \in A \Rightarrow A=b_j$\} is the finite set of \emph{actions}.
    With $\PHl^{\diamond}$ the set of all the sub-states of $\PHl$.    
      %$\PHa$ = \{ $\PHfrappe{A}{b_j}{b_k}$ with $A \in \PHl^{\diamond} \wedge b \in \PHs \wedge b_j \neq b_k \wedge$ if $b_j \in A \Rightarrow A=b_j$ \}.With $\PHl^{\diamond}$ the set of all the sub-states of $\PHl$. 
  \end{itemize}
\end{definition}

\begin{example}
The figure \ref{fig:ph} represents a $\PH$ $(\PHs,\PHl,\PHa)$ with three sorts
($\PHs = \{a, b, c\}$) and:
$\PHl_a = \{a_0, a_1\}$,
$\PHl_b = \{b_0, b_1\}$,
$\PHl_z = \{z_0, z_1, z_2\}$.
\begin{figure}[ht]
\label{fig:ph} 
\centering
\begin{tikzpicture}[apdotsimple/.style={apdot}]
%\path[use as bounding box] (0,-1) rectangle (4,4);

\TSort{(0,1)}{z}{2}{l}
\TSort{(1.5,4)}{b}{3}{t}
\TSort{(4,1)}{a}{2}{r}

\THit{b_0}{}{z_1}{.east}{z_0}
\THit{a_1}{out=60,in=0,selfhit}{a_1}{.east}{a_0}

\TActionPlur{a_0, b_1}{z_0.north east}{z_1.south east}{}{2,2}{right}

\path[bounce,bend left]
\TBounce{a_1}{}{a_0}{.north}
\TBounce{z_1}{bend left=90}{z_0}{.south east};

\TState{a_1,b_1,z_0}
\end{tikzpicture}
\caption{
A PH model example with three sorts: $a$, $b$ and $z$ ($a$ is either at level 0 or 1, $b$ at either level 0, 1 or 2 and $z$ at either level 0, or 1). Boxes represent the \emph{sorts} (network components), circles represent the \emph{processes} (component levels), and the 3 \emph{actions} that model the dynamic behavior are depicted by pairs of arrows in solid and dotted lines. A self-action:  $\PHfrappe{a_1}{a_1}{a_0}$, a mono-action:  $\PHfrappe{b_0}{z_1}{z_0}$ and plural action  $\PHfrappe{a_0 \wedge b_1}{z_0}{z_1}$.  The grayed processes stand for the possible initial state: $\PHstate{a_1, b_1, z_0}$.
}
\end{figure}
\end{example}
A state of a given PH consists in a set of active processes containing a single process of each sort.
The active process of a given sort $a \in \PHs$ in a state $s \in \PHl$
is noted $\PHget{s}{a}$.
For any given process $a_i$ we also note: $a_i \in s$ if and only if $\PHget{s}{a} = a_i$. The dynamic of the PH networks is performed thanks to the actions. Indeed, the transition from one state $s_1$ to its successor $s_2$ is done when there is a playable action (definition \ref{def:playableAction}) at $s_1$. After each transition only one sort, or one component, changes its level from one process to another.
%
%In some cases it is necessary to represent a reaction of a set of components on one component. For example in the bio-chemical reactions :$X \xrightarrow{Y} Z$ or  $X + Y \rightarrow Y + Z$, where $ X $ is a set of reactives, $ Y $ a set of catalysts and $ Z $ a set of products. % In PH network the evolution is asynchronous so we consider that $Z$ is one component. 
%Plural actions permit to represent this kind of reactions in PH. The plural is made up of two sets of processes of different sorts, which represent all the hitters and the bounces. For example the bio-chemical reaction above would be presented in PH by:  $\PHfrappe{X_1 \wedge Y_1}{Z_0}{Z_1}$, if we consider $X$, $Y$ and $Z$ are the sorts, each one is either at level 0 (if it is abscent) or 1 (if it is present).

In some dynamics it is crucial to have information about the delays between two events (two states of a PH). Classic actions cannot exhibit this information: we just know chronology, i.e., that the state $s_2$ will be after $s_1$ in the next step but it is not possible to know chronometry, i.e., how much time this transition takes to occur. We propose to add the delay in the action attributes which is responsable of the transition between the two states. That means that this action needs to be played during a specific time so that the system does not change its state (\pref{def:TimedAction}).

\begin{definition}[Timed action]
\label{def:TimedAction}
Let $\PH = (\PHs,\PHl,\PHa)$ be a process hitting and $\PHl^{\diamond}$ be the set of all the sub-states of $\PHl$.
A timed action of $\PH$ is a action with a delay $D$: $\PHfrappedelay{A}{D}{b_i}{b_j}$ where $D \in \mathds{R}^+$, $A \in \PHl^{\diamond}$, and $b_i$, $b_j$ where  $b_i \not = b_j$ are two processes of the target sort $b$. If $b_i \in A$, $A=b_i$.
\end{definition}
To model biological networks, we used the PH framework with timed actions (\pref{def:PH-timed}). Indeed, in the biological models that we studied we need to present not only the cooperation between the components to influence another one (action), but also the time that this action need to take place (delays). 
\begin{definition}[Process Hitting with Timed Actions]
\label{def:PH-timed}
  A \emph{Process Hitting with timed actions} is a triple $(\PHs,\PHl,\PHap)$ where:
  \begin{itemize}
    \item  $\PHs = \{a,b,\dots\}$ is the finite set of \emph{sorts};
    \item  $\PHl = \prod_{a\in\PHs} \PHl_a$ is the set of \emph{states} where
      $\PHl_a = \{a_0,\dots,a_{l_a}\}$
      is the finite set of \emph{processes} of sort $a\in\Sigma$
      and $l_a$ is a positive integer, with $a\neq b\Rightarrow \PHl_a \cap \PHl_b = \emptyset$;
    \item  $\PHap = \{ \PHfrappedelay{A}{D}{b_j}{b_k}  \mid A \in \PHl^{\diamond}, b_j\neq b_k, b_i \in A \Rightarrow A=b_j\}$
      is the finite set of \emph{timed actions}.
  \end{itemize}
\end{definition}

Duration of actions can now be represented in a Process Hitting model thanks to timed actions.
Note that if all actions delays are set to 0 it is equivalent to Process Hitting without delays (original PH).
The way these new actions should be used is described as follows.

\begin{definition} [Playable timed action]
\label{def:playableAction}
Let $\PH = (\PHs,\PHl,\PHap)$ be a PH with timed actions and $s \in \PHl$ a state of $PH$.
We say that the action $h = \PHfrappedelay{A}{D}{b_i}{b_j}$, with $D \geq 0 $,
is \emph{playable in a state $s$} if and only if
$A \subseteq s$ and $b_i \in s$ (\ie$ \forall a_i \in A, \PHget{s}{a} = a_i$ and $\PHget{s}{b}=b_j$).
\end{definition}

%\begin{definition}[Autonomous temporized sort]
%\label{def:TempSort}
%A sort $a$ is said to be an {\emph autonomous temporized sort} if and only if $\forall h = \PHfrappedelay{A}{D}{a_i}{a_j} \in \PHa$  where $a_i$ and $a_j$ are processes of $a$, we have only $A = \{a_i\}$.
%%it has only self actions during known delays:  $\PHfrappedelay{a_i}{D}{a_i}{a_j}$ and $\PHfrappedelay{a_j}{D'}{a_j}{a_k}$ ...
%\end{definition}
%
%The delays caused by those timed actions and temporized sorts have an impact on the system dynamics.
%Indeed, in our asynchronous model, only one action can be processing and the state of the system must not change during this time.
%But, clocks represented by autonomous temporized sorts can evolve in parallel of processing actions.
%Those new sorts are an exception to the asynchronicity behavior of our model: multiple temporized sorts can change there processes at the same time.
%Thus, making possible that a change of the state of the system may occurs during the processing of an action.
%If this case occurs, the processing of the action is stopped and a new action can be played.
%The dynamic of a Process Hitting with timed plural actions is formally defined as follows.
%
%%Ajouter l'exemple de la sorte L de l'horloge circadienne
%We define the semantics of \emph{dense-time} Process Hitting as a timed transition system.
%In this model, two kinds of transitions may occur: \emph{dense-time} transitions when time passes and \emph{discrete} transitions when a transition of the net is fired.
%
%
We denote the resulting state of the $\PH$ model ($\PH = (\PHs,\PHl,\PHap)$) after playing an action $h$ in a state $s$ is called a \emph{successor} of $s$ and is denoted by $(s \play h)$ as well as the resulting state of a sort $a \in \PHs$ by $\PHget{s}{a}$. The dynamic of the Process Hitting is based on asynchronized evolution so we have: $$\PHget{(s \play h)}{b} = b_j \text{ and } \forall c \in \PHs, c \neq b \Rightarrow \PHget{(s \play h)}{c}=\PHget{s}{c}$$.

\begin{definition}[Semantics of a Process Hitting with Timed Actions]
\label{def:semantic}
Let $\PH = (\PHs,\PHl,\PHap)$ be a dense-time Process Hitting and $\textsf{state}(\PH, t)=s$ be the state of $\PH$ at time $t$ with $s \in \PHl$. Let $h = \PHfrappedelay{A}{D}{b_i}{b_j}$ be a playable timed action at $t$ in $s$, with $D \geq 0 $ and $t, t'$ and ($t+D$) time steps. So the semantic is explained by:
$$ \forall t' \in [t, t+D ] : \textsf{state}(\PH, t')=s \text{ and } \textsf{state}(\PH, t+D)=(s \play h).$$
\end{definition}

%If $\PH = (\PHs,\PHl,\PHap)$ is a dense-time Process Hitting and $s \in \PHl$ be the state of $\PH$ at time $t$ .
We propose at \pref{def:semantic}, the semantic of the timed PH in which we base on to generate the corresponding model to the given observations. Indeed it means that if there is no action in $\PHa$ that is playable in $s$ the state of the system remains $s$ for all time $t'$ with $t' > t$.% (i.e. steady state).
If $h = \PHfrappedelay{A}{D}{b_i}{b_j}$ is the choosen playable action in $s$ then the state of $\PH$ is $s$ during $D$ time steps and at the time step $t+D$
the \emph{successor} of $s$ denoted by $(s \play h)$ is obtained.

Even if there already exists a few hybrid formalisms, we chose to propose this extension of the PH framework for several reasons.
First, PH is a general framework that,
although it was mainly used for biological networks,
allows to represent any kind of dynamical models,
and converters to several other representations are available. % (see section~\ref{comparison}).
Although an efficient dynamical analysis already exists for this framework,
based on an approximation of the dynamics,
it is interesting to identify its limits (especially the fact that previous studies were focusing only on discrete, not timed, dynamics)
and compare them to the approach we present later in this paper.
Finally, the particular form of the actions in a  PH model allows
to easily represent them in ASP,
with one fact per action, as described in the next section.
\textcolor{red}{COMMENT Tony: toujours un argument pour TACAS l'ASP ?}
Other representations may have required supplementary complexity;
for instance, a labeling would be required
if actions could be triggered by a variable number of processes.
%Parler de l'asynchrone model
We now show how to represent the previous definitions through ASP.
% and provide a case study on the example of Circadian Clock network. Later we propose an approach to resolve the completion problem of PH networks with the use of ASP.



  \input{parts/Generation-PH}
  % Explication des étapes de l'alogorithme par un exemple d'Evaluation:

\subsection{Case study}
\label{sec:case_study}

In this section we demonstrate how this method generates a PH model coherent with the set of biological regulatory time series data given as an input. 
First, the method uses discritised observations as an input (\ie chronogram), thus it is necessary to use an other method which transforms the analogic time series data to discritised time series data.

%\begin{figure}[htb!]
%Observations:\\
%\includegraphics[ width =0.35\linewidth]{images/courbes/gene-a.pdf}\\
%\includegraphics[ width =0.35\linewidth]{images/courbes/gene-b.pdf}\\
%\includegraphics[ width =0.35\linewidth]{images/courbes/gene-z.pdf}
%\hspace{0.01cm}
%\textcolor{red}{$\Rightarrow$}
%\hspace{0.01cm}
%\begin{minipage}[t]{0.4\linewidth}
%\vspace{-5cm}
%Process Hitting:
%\includegraphics[width =1\linewidth]{images/PH-but.pdf}
%\end{minipage}
%\end{figure}


% Discretization figure
\begin{figure}[h]\centering
\includegraphics[width =0.31\linewidth]{images/courbes/gene-a.pdf}
\includegraphics[width =0.31\linewidth]{images/courbes/gene-b.pdf}
\includegraphics[width =0.31\linewidth]{images/courbes/gene-z.pdf}

\includegraphics[width =0.31\linewidth]{images/courbes/gene-a-disc-change.pdf}
\includegraphics[width =0.31\linewidth]{images/courbes/gene-b-disc-change.pdf}
\includegraphics[width =0.31\linewidth]{images/courbes/gene-z-disc-change.pdf}
\caption{Examples of the discretization of continous time series data into bi-valued chronograms.
Abscisse represents time and ordinate the gene expression level.
The expression level is discretized according to a threshold fixed to the half of the gene expression value in this example. A \texttt{change(t)} is a time step in which we observe a change in the gene expression level. }
\label{fig:discretization}
\end{figure}

We can summarize our method in the following steps:
\begin{itemize}
\item[-] Detection of changes
\item[-] Computation of the interactions possibly reponsible of thoses changes
\item[-] Filtering of the candidates actions
\item[-] Add actions to the model: generation/revision of the model
\end{itemize}

% Running example

We will now show an example of the execution of the Algorithm \pref{alg:PHG_ap} on the chronograms of the observed data of \pref{fig:discretization}:

%Let $i$ be the maximal number of hitters in an action of a PH: {\textit the indegree}.

The first change occurs at $t_1$ = $t_{min}$ = 2,
wich we will denote as \texttt{change(2)}. It is the gene $"z"$ whose value changes from $0$ to $1$, thus the action that has realized this change is of the form $h = \PHfrappedelay{A}{D}{z_0}{z_1}$,
where $ A \in \PHl^{\diamond}, | A| \leq i$, (we suppose $i=2$) and $D$ is the delay wich is equal to $2$ here, since
$D_{t_i}=t_i - t_{i-1}$, such that $\exists$ \texttt{change($t_i$)} and \texttt{change($t_{i-1}$)},
$D_{t_1}= t_1 - t_0 = 2 - 0 = 2$.

Let $R=\{ b \rightarrow z, a \rightarrow z, a \rightarrow a \}$
be the set of regulation influences amoung the components of the system.
%
In the first change $t_1$ = 2 that we will denotes as $change(2)$,
It has been caused by an action $h = \PHfrappedelay{A}{2}{z_0}{z_1}$. According to $R$ the genes having influence $"z"$ are $G_{R_z} = \{a, b\}$. It means that $A= \{ a_{\textcolor{red}{?}}, b_{\textcolor{red}{?}} \} $ or $A= \{ a_{\textcolor{red}{?}} \} $ or $A= \{ b_{\textcolor{red}{?}} \} $.
%
The expression level of the genes of $G{R_z}$ between $t_i$ and $t_{i-1}$ is computed from the observations as follows:
\begin{itemize}
\item[-] $a \in  G{R_z}$: $[a]_t=0$ $\forall t \in [0,2] $
\item[-] $b \in  G{R_z}$: $[b]_t=1$ $\forall t \in [0,2] $
\end{itemize}
%
Thus $A= \{ a_0, b_1 \} $ or $A= \{ a_0\} $ or $A= \{ b_1 \} $ and the set of candidate actions is:
$H_{change(2)} = \{ h_1=\PHfrappedelay{a_0}{2}{z_0}{z_1}
, h_2=\PHfrappedelay{b_1}{2}{z_0}{z_1}
, h_3=\PHfrappedelay{a_0 \wedge b_1 }{2}{z_0}{z_1} \}$.

The second change occurs at $t_2 = 3$ and we will denote it as $change(3)$.
Here its the gene $"a"$ whose value changes from $a_0$ to $a_1$, thus the action that has realized this change is of the form
$h = \PHfrappedelay{A}{D}{a_0}{a_1}$ 
where $ A \in \PHl^{\diamond}, | A| \leq 2$ and $D$ is the delay that is equal to $1$ here: 
$D_{t_2}= t_2 - t_1 = 3 - 2= 1$.
According to $R$ the genes which influence $"a"$ are $G_{R_a} = \{a\}$.
It means that $A= \{ a_{\textcolor{red}{?}}\}$ and the expression level of $"a"$ between $t_1$ and $t_{2}$ is $a_0$.
So  $A= \{ a_0\} $ and it is an auto-influence. Thus there is only one candidate action that is a \emph{self-action}:
$H_{change(3)} = \{ h=\PHfrappedelay{a_0}{1}{a_0}{a_1}  \}$.

The third change occurs at $t_3 = 4$ and we will denote it as $change(4)$.
Here it is the gene $"b"$ whose value changes from $b_1$ to $b_0$, thus the action that has realized this change is of the form 
$h = \PHfrappedelay{A}{D}{b_1}{b_0}$ where $ A \in \PHl^{\diamond}, | A| \leq 2$ and $D$ is the delay and it is equal to $1$ ($D_{t_3}= t_3 - t_2 = 4 - 3 = 1$).
According to $R$ there is no gene that can influences $"b"$, thus no action can realizes this change.

The fourth change occurs at $t_4 = 5$ and we will denote it as $change(5)$.
Here it is $"a"$ whose value changes from $a_1$ to $a_0$, thus the action that has realized this change is of the form $h = \PHfrappedelay{A}{D}{a_1}{a_0}$ \\
where $ A \in \PHl^{\diamond}, | A| \leq 2$ and $D$ is the delay that is equal to $1$ ($D_{t_4}= t_4 - t_3 = 5 - 4= 1$).
According to $R$ the genes which influence $"a"$ are $G_{R_a} = \{a\}$.
Again $A= \{ a_{\textcolor{red}{?}}\}$ and since the expression level of $"a"$ between $t_3$ and $t_{4}$ is $a_1$,
we have $A= \{ a_1\} $ and there is only one candidate action that is a \emph{self-action}
$H_{change(5)} = \{ h=\PHfrappedelay{a_1}{1}{a_1}{a_0} \}$.

The fifth change occurs at $t_5 = 6$ and we will denote it as $change(6)$.
Here it is $"z"$ whose value changes from $z_1$ to $z_0$, thus the action that has realized this change has the form of:
$h = \PHfrappedelay{A}{D}{z_1}{z_0}$ \\
where $ A \in \PHl^{\diamond}, |A| \leq 2$ and $D$ is equal to 1 here:
$D_{t_5}= t_5 - t_4 = 6 - 5= 1$ \\
According to $R$, $G_{R_z} = \{a, b\}$.
It means that $A= \{ a_{\textcolor{red}{?}}, b_{\textcolor{red}{?}} \} $ or $A= \{ a_{\textcolor{red}{?}} \} $ or $A= \{ b_{\textcolor{red}{?}} \} $
The expression level of $"a"$ and $"b"$ between $t_4$ and $t_{5}$ is respectively $a_0$ and $b_0$.
Thus $A= \{ a_0, b_1 \} $ or $A= \{ a_0\} $ or $A= \{ b_0 \} $ \\
The candidates action are:
$H_{change(6)} = \{ h_1=\PHfrappedelay{a_0}{1}{z_1}{z_0}$, $  h_2=\PHfrappedelay{b_0}{1}{z_1}{z_0}$, $  h_3=\PHfrappedelay{a_0 \wedge b_0 }{1}{z_1}{z_0} \}$.

After proccessing all chronograms, the candidate actions are: \\
$H_{change(2)} = \{ h_1=\PHfrappedelay{a_0}{2}{z_0}{z_1},  h_2=\PHfrappedelay{b_1}{2}{z_0}{z_1}, h_3=\PHfrappedelay{a_0 \wedge b_1 }{2}{z_0}{z_1} \}$.\\
$H_{change(3)} = \{ h_4=\PHfrappedelay{a_0}{1}{a_0}{a_1}  \}$. \\
$H_{change(5)} = \{ h_5=\PHfrappedelay{a_1}{1}{a_1}{a_0}  \}$. \\
$H_{change(6)} = \{ h_6=\PHfrappedelay{a_0}{1}{z_1}{z_0}$ , $  h_7=\PHfrappedelay{b_0}{1}{z_1}{z_0}, h_8=\PHfrappedelay{a_0 \wedge b_0 }{1}{z_1}{z_0} \}$. \\

At this stage of the process, all candidates actions are consistent with all observations and given regulation influences.
Until now the method used ensured completeness: here we have the complete set of consistent action that can explain the observations.



  \section{Refinement of a genrated PH model }
The Algorithm \ref{alg:PHG_ap} return a PH model with a set of actions compeletely coherent with the observations as well as the regulatory genes influences.
But in practice this set of actions can be further refined using a background knowledge.
The following filtering operation can help to reduce the complexity of the model learned/revised.

\subsection{$1^{st}$ filter: priority on given model}

If we want to minimize the number of actions added to the input PH we can consider that the action of this PH are more trustable than the generated one. Thus generated actions that only explain changes that can already be explain by given action can be discarded: $\forall t \in T$ such that $\exists$  \texttt{change(t)}, we have:
\begin{itemize}
\item[-] \texttt{$H_c$}: the set of candidates action generated to explain $change(t)$
\item[-] \texttt{$H_{ini}$:} the set of actions of $PH_{ini}$ that can realize the change at $t$
\item[-] \texttt{$H_{final}$:} the set of actions to keep as to explain the change at $t$
\end{itemize}
We have either:
\begin{itemize}
\item[•] If $H_c \cap H_{ini} \neq \emptyset $ then $H_{final}= H_c \cap H_{ini}$ 
\item[•] If $H_c \cap H_{ini} = \emptyset $ then $H_{final}=H_c$
\end{itemize}
In practice, the change that already can be explained by the input PH can be detected before computing the candidates actions, allowing us to discard them without generating them.
If in our running example,
we start with $H_{ini}=\{h_ini\PHfrappedelay{a_0}{1}{z_1}{z_0}\}$,
since the action $h_ini$ can realize $change(6)$ there is no need to generate new ones.
Here, $H_{change(6)}$ will only consist of $h_ini = h_6$, $h_7=\PHfrappedelay{b_0}{1}{z_1}{z_0}$ and $h_8=\PHfrappedelay{a_0 \wedge b_0 }{1}{z_1}{z_0} $ will be discarded. 

\subsection{$2^{nd}$ filter: strict influences (activator or inhibitor)}

If knowledge about strict influences is given it can be used to discarded actions that are conflicting with those influences.
For example, if we know that a gene $a$ influences a gene $b$ and that $a$ can only inhibit $b$, then the actions using $a$ as hitter to increase the value of $b$ are inconsistent and can be discarded.
$\forall h \in H$ such $h=\PHfrappedelay{A}{D}{b_n}{b_m}$, with $ A \in \PHl^{\diamond}, | A| \leq i$ ($i$: indegree of the algorithm) we have :

\begin{itemize}
\item[•] If $n < m$, if $\exists G_k \in \PHl_G $ such that $G_k \xrightarrow{(-)} b$ (and $\nexists G_k \xrightarrow{(+)} b$ ) and $k \neq 0$ then $h$ can be discarded.
\item[•] If $n > m$, if $\exists G_k \in \PHl_G $ such that $G_k \xrightarrow{(+)} b$ (and $\nexists G_k \xrightarrow{(-)} b$ ) and $k \neq 0$ then $h$ can be discarded.
\end{itemize} 

In our running example, if we now that $b$ is an inhibitor of $z$, all actions where $a$ is used to activate $z$ can be discarded.
Here, we have an observation where $z$ is activated in $change(2)$ where its value switches from $z_0$ to $z_1$.
We now that $a$ as an influence on $z$ but as an inhibitor not as an activator thus the actions $h_2=\PHfrappedelay{b_1}{2}{z_0}{z_1}$ and $h_3=\PHfrappedelay{a_0 \wedge b_1 }{2}{z_0}{z_1}$ will be discarded.

\subsection{$3^rd$ filter: delay merging}

When the time information of observation is not perfect, the same regulation interaction may append with different delay.
One simple solution to deal with such input can be to simply agregate action that differ only by their delay.
We can merge each action with the same hitters, $S_1,P_1,\ldots, S_n,P_n$ and the same target, $G, P, P'$, into one action where the delay is the average.
$\forall h_1, h_2,..., h_k \in H$ such that $h_1=\PHfrappedelay{A}{D_1}{a_n}{a_m}$, $h_2=\PHfrappedelay{A}{D_2}{a_n}{a_m}$, ..., $h_k=\PHfrappedelay{A}{D_k}{a_n}{a_m}$ with $ A \in \PHl^{\diamond}$, $a_n, a_m \in \PHl_a$ et $D_1 \neq D_2 \neq ... \neq D_k$ then : \\
\texttt{fusion} all actions $h_1, h_2,..., h_k$ into one action $h$ $$h=\PHfrappedelay{A}{D_{average}}{a_n}{a_m}$$ such that: 
$$D_{average} = \frac{\sum_{i=1}^k D_{i}}{k} $$

For example, lets suppose that we came out with two actions $h=\PHfrappedelay{a_0}{2}{z_0}{z_1}$ and $h'=\PHfrappedelay{a_0 \wedge b_1 }{4}{z_0}{z_1}$, they will merged into $h_2=\PHfrappedelay{b_1}{3}{z_0}{z_1}$.
If input data are perfect, usage of this merging is totally unsafe and can lead to a set of actions that cannot produce any of the observed changes in the worst case.
The intuition behind this method is to give a first idea of how to cope with big amount of real data which are not perfect.
The idea is that if enough observations are provided, the delay of the action will be more precise.


  \section{Evaluation}
	\label{sec:evaluation}
	
	In this section, we assess the efficiency of our algorithm through case studies coming from the DREAM4 challenge \cite{prill2011crowdsourcing}.

	DREAM challenges are annual reverse engineering challenges that provide biological case studies.
	In this paper, we focus on the datasets coming from DREAM4.
	The input data that we tackle here consists of the following:
	5 different systems each composed of 10 genes, all coming from E. coli and yeast networks. For every such system,
	the available data are the following: (i) 5 time series data with 21 time points; (ii) steady state at wild type;
	(iii) steady states after knocking out each gene;
	(iv) steady states after knocking down each gene (i.e. forcing its transcription rate at 50\%);
	(v) steady states after some random multifactorial perturbations. We processed all the data.
	Here, we focus on the management of time series data.

\subsection{Settings}

	Time series data provide us 20 transitions.
	Each of them include different perturbations that are maintained all time along during the first 10 transitions and applied to at most 3 genes.
	In this setting, a perturbation means a significant increase or decrease of the gene expression.
	%
	In the raw data of the time series, gene expression values are given as real number between 0 and 1.
	To apply our approach, we chose to discretize those data into 4 qualitative values.
	Each gene is discretized in an independent manner, with respect to the following procedure:
	we compute the average value of the gene expression among all data of a time series,
	then the values between the average and the maximal/minimal value are divided into as many levels.
	Discretizing the data according to the average value of expression is expected to reduce the impact of perturbation on the discretization and thus on the model learned.

\subsection{Results}

  \section{Conclusion and perspectives}
\label{sec:conclusion}

In this paper, we proposed an approach to automatically infer timed models of Process Hitting from time series data (expressed as chronograms). To do so, we implemented our algorithm in ASP. We illustrated the applicability and limits of the method through various benchmarks. This opens the way to promising applications in the connection between biologists and computer scientists. Further works will now consist in discussing the kind of information one can get on timed Process Hitting by analyzing the associated untimed model. We also plan to improve our implementation to make it robust against noisy data.  

%Nous avons montré dans cette thèse de Master une nouvelle analyse dynamique développée. Cette analyse est applicable à une classe de modèles dits PH et vise à déterminer des propriétés des réseaux modélisés. 

%La première propriété est la recherche des états stables du réseau qu'on appelle les points fixes. Ces points représentent les états du réseau pendant lesquels, le modèle ne peut plus évoluer. Il est intéressant à les connaitre car ils bloquent l'évolution des systèmes biologiques. Nous avons développé deux méthodes, sachant que la deuxième est plus optimale, qui retournent l'ensemble de tous les points fixes du réseau. Ce point fixe n'est qu'un état du réseau traduit en ASP par un niveau pour chaque composant, autrement un processus pour chaque sorte. \\
%La deuxième propriété est une propriété qui se base sur la dynamique du réseau: l'atteignabilité. Un réseau biologique évolue dans plusieurs sens qui peuvent aboutir ou pas à des états-objectifs. Notre nouvelle approche retourne les chemins exactes aboutissant à atteindre un niveau cible d'un composant du système. Ce chemin se traduit par un ensemble de changements successives des niveaux. En PH cela se traduit par de changements successifs de processus et c'est ce que nos méthodes retournent comme résultat. Il s'avére que la méthode itérative en ASP est plus optimale que celle en ASP normal. En effet il n'est pas nécessaire de prévoir le nombre de changements pour la méthode itérative et elle retourne le résultat plus rapidement (quelques seconde pour des réseau moyennement grand).

%Une comparaison a été faite par rapport à l'existant, \textsc{PINT} et la méthode de Rocca et al. Les résultats montrent que, par rapport à \textsc{PINT}, la méthode de recherche des points fixes est efficace, mais que pour l'accessibilité, elle l'est moins que prévu. Cependant aussi notre méthode retourne de plus le chemin d'atteignabilité. Elle offre la possibilité de poser des questions plus générales par rapport à \textsc{PINT} portant sur plusieurs sortes de plus. \\
%Sachant que la présentation d'un réseau biologique en PH est simplifie le traitement et la traduction des modèles, il s'avère que notre méthode qui se base sur ce formalisme est plus efficace que d'autres méthodes développées en ASP aussi mais pour des réseau de graphes de transitions. Le cas de la méthode de Rocca qui est gourmande en temps par rapport à la notre, résultat retourné en des minutes contre un résultat affiché en quelques secondes.\\

%Nous pensons que cette approche peut également être utilisée et adaptée avec d'autres modèles tels que le modèle de Thomas, les réseaux de Petri et les modèles synchrones. Cela nécessite une traduction propre au modèle étudié ainsi qu'un traitement approprié.\\
%Parmi nos perspectives qui font partie de mon sujet de thèse, c'est d'essayer d'améliorer cette méthode en éliminant les cycles de la méthode itérative. Cela évite de tourner indéfiniment dans des boucles sans avoir un résultat affiché.\\ 

%Ensuite, nous souhaitons étendre le programme pour chercher les attracteurs. Un attracteur est un ensemble d'états à partir desquels il n'est plus possible de sortir, et donc tel que le réseau tourne indéfiniment dans ces états. Le point fixe est un cas spécial des attracteur, en effet c'est un attracteur de dimension une. Par contre, la caractérisation des attracteurs  de la dynamique,de dimension $n$, requiert une analyse des dynamiques possibles bien plus poussée que pour les points fixes.\\
%Nous visons aussi à implémenter une recherche dynamique dans le sens inverse de l'atteignabilité et poser la question: "\textit{Quels sont les états initiaux qui nous permettent d'atteindre nos objectifs?}".  La réponse à cette question est l'ensemble des états à partir des quels il existe des chemins qui activent le ou les objectif(s). C'est vrai que la méthode de Rocca et al. résolve cette problématique mais nous estimons à avoir une approche plus efficace en terme de temps et qui utilise le réseau en process hitting en non pas les graphes de transitions.

%Tous ces problématiques constituent une perspective intéressante dans le cadre du développement de techniques d'analyse statique et dynamiques des propriétés du Process Hitting.\\
\section*{Acknowledgment}
This paper is based upon work supported by the Agence Nationale de la Recherche for the HyClock project under Grant No. ANR-14-CE09-0011.

  \bibliographystyle{plain}
  \bibliography{biblio}
%\section*{Appendix: Springer-Author Discount}

\end{document}
